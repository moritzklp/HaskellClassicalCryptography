\documentclass[12pt,a4paper]{article}
\input{latexmacros.tex}

\title{Classical Cryptography Report}
\author{Eva Imbens, Chiara Michelutti, Jan Przystał, Moritz Klopstock}
\date{\today}
\hypersetup{pdfauthor={Me}, pdftitle={My Report}}

\begin{document}

\maketitle

\begin{abstract}
    This project presents a comprehensive exploration of classical cryptographic techniques using Haskell. It covers the design, implementation, and analysis of three historical ciphers: the Caesar cipher, the Vigenère cipher, and the One-Time Pad (OTP). The report details the development of haskell functions for encryption and decryption, employing frequency analysis and statistical methods to crack ciphers and expose vulnerabilities, such as the Many-Time Pad attack on OTP. By leveraging Haskell’s strong type system, lazy evaluation, and performance optimizations, the project illustrates both the theoretical security and practical limitations of these cryptographic systems.
\end{abstract}

\vfill

\tableofcontents

\clearpage

% We include one file for each section. The ones containing code should
% be called something.lhs and also mentioned in the .cabal file.

\section{Introduction}
In this project\footnote{\url{https://github.com/moritzklp/FP-ProjectTP}}, 
we explore fundamental principles of classical cryptography by implementing three historical ciphers: Caesar, Vigenère, and One-Time Pad (OTP). 
Alongside implementing these encryption techniques, we also investigate their weaknesses and known cryptographic attacks. 
Specifically, we focus on:

\begin{itemize}
    \item Frequency analysis on the Caesar cipher (and monoalphabetic substitutions in general), 
    showing how statistical patterns in natural language can reveal encrypted messages.
    \item Kasiski examination and Friedmann tests on Vigenère, 
    which exploits repeating patterns in the ciphertext to determine the key length and ultimately decrypt the message.
    \item Many-Time Pad (MTP) attacks on OTP, demonstrating how key reuse undermines its security.
\end{itemize}

Our approach includes encrypting and decrypting natural language messages, generating secure random keys, 
and implementing attacks to expose vulnerabilities in these ciphers. 
By doing so, we aim to highlight the contrast between theoretical security (as in OTP) 
and practical weaknesses (as in cases of poor key management and cipher design).

The following sections outline our project’s objectives, methodologies, and experimental setup, 
detailing our implementation and security analysis in a Haskell-based environment.


\subsection{Haskell Background}
\label{sec:why_haskell}
Haskell offers several advantages that make it a strong candidate for implementing classical cryptographic algorithms. 
These benefits include performance, memory safety, and a strong type system. 
They all contribute to writing secure and reliable (cryptographic) code. 

\paragraph{Compiled and Optimized Execution} 
Despite being a high-level functional language, Haskell can perform nearly as well as C. 
Research has shown that cryptographic functions implemented in Haskell can perform within the same order of magnitude as C, 
particularly when using compiler optimizations \cite{tevis2006secure}. 
This proves that it can handle computationally intensive cryptographic tasks, with the correct optimizations.

\paragraph{Lazy Evaluation} 
Haskell uses lazy evaluation, which means that it only computes the values when they are needed. 
This can help to improve the efficiency by avoiding unnecessary calculations. 
For our many-time pad attack this would help to handle large ciphertexts efficiently by processing them only when required. 
This also helps to reduce the memory usage and computation overhead.
Lazy Evaluation may however be a security concern. It can lead to timing attacks which may leak sensitive information \cite{lazy2013}.

\paragraph{Memory Safety}
Unlike languages like C, Haskell automatically manages the memory, preventing vulnerabilities such as buffer overflows and pointer-related bugs. 
This automatic memory management ensures that cryptographic operations do not suffer from unintended memory corruption. 
This is important in cryptographic applications because small memory errors can lead to security flaws. 

\paragraph{Strong Type System}
Haskell has a strong type system that ensures that variables hold only correct kinds of values. 
This prevents unintended operations, such as treating a byte array as a string, misinterpreting cryptographic data formats. 
Programming on a type-level allows to encode security properties at compile time, which ensures that many classes of bugs are detected early.

\paragraph{Immutability}
Haskell's immutability ensures that once a value is assigned, it cannot be altered. 
This prevents unintended modifications of cryptographic data during the execution, 
which can be a problem in other languages where variables can be overwritten accidentally. 
Since cryptographic attacks and defenses often rely on maintaining strict data integrity, 
immutability provides a significant security advantage.

\paragraph{Arbitrary Precision Arithmetic}
Haskell provides arbitrary precision integers, 
which means that it allows computations with arbitrary large numbers which prevents the overflow issues that are common in many other languages. 
This is useful in cryptographic application where calculations may involve large integers, and unexpected overflows could lead to incorrect results.

\paragraph{Purity}
Haskell's pure functions make sure that the same input always produces the same output, 
which makes computations easier to test and debug. The lack of hidden side effects simplifies formal reasoning about cryptographic operations, 
which is useful in security audits and verification processes.



\input{lib/Frequency.lhs}

\input{lib/CaesarCipher.lhs}

\input{lib/VigenereCipher.lhs}

\input{lib/Pad.lhs}

\input{lib/OTP.lhs}

\input{lib/MTP.lhs}

\input{exec/Main.lhs}


\section{Conclusion}\label{sec:Conclusion}

Our project successfully implemented encryption, decryption, and key generation capabilities for three classical ciphers: 
Caesar, Vigenère, and One-Time Pad. 
The Haskell implementation includes automated attacks for each cipher, enabling users to recover plaintext from ciphertexts through methods such as 
brute force attacks, frequency analysis, Kasiski examination, Friedman tests, and the Many-Time Pad attack.

The comprehensive user interface provides access to all implemented cryptographic operations and attack methods. 
Throughout this project, we observed a good alignment between Haskell's functional programming paradigm and the implementation of classical ciphers. 
The fact that classical ciphers are simply operations that transform inputs to outputs without maintaining state, 
pairs elegantly with Haskell's functional approach. 
Each cipher could be modeled as a series of pure functions, making the code both easy to understand 
and closely resembling the mathematical descriptions of the cryptographic algorithms.

\subsection{Future Work}
Our program has several opportunities for improvement. 
A priority would be implementing more sophisticated word prediction algorithms to complete partially recovered plaintexts, 
particularly when using the Many-Time Pad attack. This would increase the effectiveness of our cryptanalysis tools.

It would also be interesting to compare the performance of the Haskell implementations of the
ciphers to other programming languages like C or Java. Cryptographic operations often need to
be performed quickly in resource constrained systems so it would be interesting to see how well
Haskell performs in such scenarios.


\newpage

\addcontentsline{toc}{section}{Bibliography}
\bibliographystyle{alpha}
\bibliography{references.bib}

\end{document}
