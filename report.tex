\documentclass[12pt,a4paper]{article}
\input{latexmacros.tex}

\title{Classical Cryptography Report}
\author{Eva Imbens, Chiara Michelutti, Jan Przystał, Moritz Klopstock}
\date{\today}
\hypersetup{pdfauthor={Me}, pdftitle={My Report}}

\begin{document}

\maketitle

\begin{abstract}
    This project presents a comprehensive exploration of classical cryptographic techniques using Haskell. It covers the design, implementation, and analysis of three historical ciphers: the Caesar cipher, the Vigenère cipher, and the One-Time Pad (OTP). The report details the development of haskell functions for encryption and decryption, employing frequency analysis and statistical methods to crack ciphers and expose vulnerabilities, such as the Many-Time Pad attack on OTP. By leveraging Haskell’s strong type system, lazy evaluation, and performance optimizations, the project illustrates both the theoretical security and practical limitations of these cryptographic systems.
\end{abstract}

\vfill

\tableofcontents

\clearpage

% We include one file for each section. The ones containing code should
% be called something.lhs and also mentioned in the .cabal file.

\section{Introduction}
In this project, we explore fundamental principles of classical cryptography by implementing three historical ciphers: One-Time Pad (OTP), Caesar, and Vigenère. 
Alongside implementing these encryption techniques, we also investigate their weaknesses and known cryptographic attacks. 
Specifically, we focus on:

\begin{itemize}
    \item Many-Time Pad (MTP) attacks on OTP, demonstrating how key reuse undermines its security.
    \item Frequency analysis on the Caesar cipher (and monoalphabetic substitutions in general), 
    showing how statistical patterns in natural language can reveal encrypted messages.
    \item Kasiski examination (or a similar method) on Vigenère, 
    which exploits repeating patterns in the ciphertext to determine the key length and ultimately decrypt the message.
\end{itemize}

Our approach includes encrypting and decrypting natural language messages, generating secure random keys, 
and implementing attacks to expose vulnerabilities in these ciphers. 
By doing so, we aim to highlight the contrast between theoretical security (as in OTP) 
and practical weaknesses (as in cases of poor key management and cipher design).

The following sections outline our project’s objectives, methodologies, and experimental setup, 
detailing our implementation and security analysis in a Haskell-based environment.


\subsection{Why Haskell}\label{sec:why_haskell}

\paragraph{Compiled and Optimized Execution} Studies have shown that Haskell implementations of cryptographic functions can perform nearly as well as C. 
For example, a Haskell implementation of the CAST-128 cipher ran within the same order of magnitude as an equivalent C version. 
This proves that it can handle computationally intensive cryptographic tasks.

\paragraph{Lazy Evaluation} Haskell uses lazy evaluation, which means that it only computes the values when they are needed.
This can help to improve the efficiency by avoiding unnecessary calculations. 
For our many-time pad attack this would help to handle large ciphertexts efficiently by processing them only when required.
This also helps to reduce the memory usage and computation overhead.

\paragraph{Memory Safety}
Unlike languages like C, Haskell automatically manages the memory, preventing vulnerabilities such as buffer overflows and pointer-related bugs.
This is important in cryptographic applications because small memory errors can lead to security flaws. 
With Haskell we do not have to worry about memory corruption or unexpected behavior due to false memory management.

\paragraph{Strong Type System}
Haskell has a strong type system that ensures that variables hold only correct kinds of values. 
This reduces the programming mistakes such as mixing up data types or performing incorrect calculations on data.
The type system also prevents unintended operations, such as treating a byte array as a string, which could introduce security risks in cryptographic applications.

\paragraph{Immutability}
Haskell's immutability (by default) means that once a value is created, it cannot be changed. 
This prevents unintended modifications of the data during the execution, which can be a problem in other languages where variables can be overwritten accidentally.

\paragraph{Arbitrary Precision Arithmetic}
Haskell allows computations with arbitrary large numbers which prevents the overflow issues that are common in many other languages.
This is useful in cryptographic application where calculations may involve large integers, and unexpected overflows could lead to incorrect results.

\paragraph{Purity}
Haskell's pure functions make sure that the same input always produces the same output, which makes cryptographic computations easier to test and debug. 
Since there are no hidden side effects, cryptographic functions can be verified more easily than in imperative languages.



\input{lib/Pad.lhs}

\input{lib/Frequency.lhs}

\input{lib/CaesarCipher.lhs}

\input{lib/VigenereCipher.lhs}

\input{lib/OTP.lhs}

\input{lib/MTP.lhs}

\input{exec/Main.lhs}


\section{Conclusion}\label{sec:Conclusion}

We have managed to implement encryption, decryption, for the Caesar, Vigenere, and One-Time Pad ciphers in Haskell. Additionaly our program allows for random key generation for all of them. We were also able to perform automated attacks, which allowed us to correctly guess the content of ciphertexts, for all implemented ciphers using brute force, frequency analysis, Kasiski examination and Friedmann tests, and Many-Time Pad. The program also contains a user interface to perform all those actions.

We have found Haskell's functional paradigm to match very well with how cphers work. They are based on cryptograhic functions which simply take an input and provide an output. A few operations are sometimes chained together, but the state never needs to be preserved which fits perfectly with Haskell.

\subsection{Future Work}
Our program could be improved to fully guess the words from partially discovered plaintexts, especially when performing the Many-Time Pad attack.

It would also be interesting to compare the performance of the Haskell implementations of the ciphers to other programming languages like C or Java. Cryptographic operations often need to be performed quickly in resource constrained systems so it would be interesting to see how well Haskell performs in such scenarios.

Finally, we can see that \cite{liuWang2013:agentTypesHLPE} is a nice paper.


\newpage

\addcontentsline{toc}{section}{Bibliography}
\bibliographystyle{alpha}
\bibliography{references.bib}

\end{document}
